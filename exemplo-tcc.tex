%
% Exemplo LaTeX de TCC do IFSul
%
% Este documento é um exemplo de uso do modelo LaTeX de TCCs do IFSul.
% O conteúdo do documento é baseado nas normas ABNT relacionadas à
% elaboração de trabalhos acadêmicos e no Guia de Normalização do
% Instituto (http://www.pelotas.ifsul.edu.br/portal/index.php?option=com_content&view=article&id=1181&Itemid=178).
%
% Os elementos textuais abaixo são apresentados na ordem em que devem
% aparecer no documento.  Repare que nem todos são obrigatórios - isso
% é devidamente indicado em cada caso.
%
% Este documento é de domínio público.
%
\documentclass{ifsultcc}

%=======================================================================
% Dados gerais sobre o trabalho.
%=======================================================================
\author{Tal, Fulano de}
\title{Orientações para Elaboração de Trabalhos de Conclusão de Curso
do IFSul Campus Sapiranga}
\orientador{Knuth, Prof.~Dr.~Donald E.}
%\coorientador{Lamport, Prof.~Dr.~Leslie}	% opcional
%\campus{Sapiranga} % opcional
\curso{Técnico em Manutenção e Suporte em Informática}
\natureza{Trabalho de Conclusão de Curso apresentado como
requisito parcial para a obtenção do título de Técnico em Informática
pelo\\Instituto Federal Sul-Rio-Grandense.\\Área de concentração: Informática.}
%\local{Sapiranga} % opcional
\ano{2015}
\palavrachave{TCC}
\palavrachave{Modelo}
\palavrachave{\LaTeX}
\palavrachave{ABNT}
\palavrachave[english]{TCC}
\palavrachave[english]{Model}
\palavrachave[english]{\LaTeX}
\palavrachave[english]{ABNT}

%=======================================================================
% Início do documento.
%=======================================================================
\begin{document}
\capa
\folhaderosto

%=======================================================================
% Dedicatória (opcional).
%=======================================================================
%\begin{dedicatoria}
%Aos nossos pais.
%
%\textit{If I have seen farther than others,\\
%it is because I stood on the shoulders of giants.}\\
%--- \textsc{Sir Isaac Newton} % \textsc é o "small caps"
%\end{dedicatoria}

%=======================================================================
% Agradecimentos (opcional).
%=======================================================================
%\begin{agradecimentos}
%Obrigado!
%\end{agradecimentos}

%=======================================================================
% Epígrafe do início do documento (opcional).
%=======================================================================
%\begin{epigrafe}
%``\textit{Ninguém abre um livro sem que aprenda alguma coisa}''.\\
%(Anônimo)
%\end{epigrafe}

%=======================================================================
% Resumo em Português.
%
% A recomendação é para 150 a 500 palavras.
%=======================================================================
\begin{abstract}
Este documento apresenta as diretrizes para formatação de TCCs para o IFSul, Campus Sapiranga, elaboradas com base nas normas da ABNT e no Guia de Normalização do IFSul.  São apresentadas as regras de estruturação e formatação gráfica para os trabalhos, bem como algumas orientações sobre a escrita do texto em si.  O documento serve também como exemplo de aplicação dessas diretrizes --- todos os elementos de texto aqui apresentados estão formatados de acordo com as normas.  A ABNT recomenda que o Resumo, no caso de trabalhos acadêmicos, contenha de 150 a 500~palavras, e que seja utilizado parágrafo único. Deve-se usar o verbo na voz ativa e na terceira pessoa do singular.
\end{abstract}

%=======================================================================
% Resumo em língua estrangeira (obrigatório somente para teses e
% dissertações).
%
% O idioma usado aqui deve necessariamente aparecer nos parâmetros do
% \documentclass, no início do documento.
%=======================================================================
%\begin{otherlanguage}{english}
%\begin{abstract}
%This document presents guidelines on the use of IFSul's \LaTeX\ class for academic reports and dissertations.  At the same time, it serves as an example on using the class, employing the main commands and providing further general orientations on the use of \LaTeX.  In addition, we have added guidelines for the process of writing itself, collecting tips and recommendations that contribute to the technical quality enhancement of academic monographs.  The Abstract should be composed of 150 to 500~words and must not contain any citations.  It is suggested that a single paragraph be used.
%\end{abstract}
%\end{otherlanguage}

%=======================================================================
% Lista de Figuras (somente se houver figuras no texto).
%=======================================================================
\listoffigures

%=======================================================================
% Lista de Tabelas (somente se houver tabelas no texto).
%=======================================================================
\listoftables

%=======================================================================
% Lista de Abreviaturas e Siglas (opcional).
%
% Deve ser passada como parâmetro a maior das abreviaturas ou siglas
% utilizadas.
%=======================================================================
%\begin{listadeabrevsiglas}{FAPERGS}
%\item[ABNT] Associação Brasileira de Normas Técnicas
%\item[ampl.] ampliado, -a
%\item[atual.] atualizado, -a
%\item[CAPES] Coordenação de Aperfeiçoamento de Pessoal de Nível Superior
%\item[coord.] coordenador
%\item[FAPERGS] Fundação de Amparo à Pesquisa do Estado do Rio Grande do Sul
%\item[N.~T.] Novo Testamento
%\item[seg., segs.] seguinte, -s
%\end{listadeabrevsiglas}

%=======================================================================
% Lista de Símbolos (opcional).
%
% Deve ser passado o maior (mais largo) dos símbolos utilizados.
%=======================================================================
%\begin{listadesimbolos}{Ca}
%\item[\textsuperscript{o}C] Graus Celsius
%\item[Al] Alumínio
%\item[Ca] Cálcio
%\end{listadesimbolos}

%=======================================================================
% Sumário
%=======================================================================
\tableofcontents

%=======================================================================
\chapter{Apresentação Gráfica}
Este capítulo relaciona as regras para apresentação gráfica do documento.

\section{Papel e impressão}
Os trabalhos devem ser formatados em tamanho~A4 (210~mm $\times$ 297~mm).  Quando impressos, devem ser usados frente e verso das folhas.  As margens para o texto principal devem ser de:
\begin{itemize}
	\item 3~cm na borda superior;
	\item 2~cm na borda inferior;
	\item 3~cm na borda interna (lado esquerdo nas páginas ímpares e lado direito nas páginas pares);
	\item 2~cm na borda externa (lado direito nas páginas ímpares e lado esquerdo nas páginas pares).
\end{itemize}
  
\section{Formatação geral de texto}
O texto em geral deve ser formatado em fonte Times ou equivalente (ex.: Times New Roman) tamanho~12, com espaçamento 1,5 entre linhas.  Os parágrafos devem apresentar, na primeira linha, um recuo de 0,6~cm da margem esquerda.  Não deve haver espaçamento adicional entre os parágrafos.

Alguns itens de texto devem ser formatados em fonte menor, tamanho~10, e com espaçamento simples.  São eles:
\begin{itemize}
	\item legendas de figuras e tabelas;
	\item notas de rodapé;
	\item citações longas;
	\item numeração das páginas.
\end{itemize}

A lista de referências bibliográficas deve ser formatada com a fonte em tamanho~12, mas com espaçamento simples.

Palavras e expressões em língua estrangeira devem ser apresentadas em destaque, usando o estilo itálico, como em \textit{capability}.

\section{Numeração de páginas}
O número da página deve ser posicionado no topo da folha, a 2~cm da borda superior do papel, e alinhado à margem externa (à direita nas páginas ímpares e à esquerda nas páginas pares).  A numeração deve ser mostrada somente a partir do primeiro capítulo do texto.  As páginas devem ser contadas a partir da folha de rosto (ou seja, a capa não entra na contagem).  

\section{Estruturação do texto}
O texto pode ser dividido em até 5 níveis de seções. As seções primárias (capítulos) devem iniciar em nova folha (sempre em página ímpar --- se necessário, deixe uma página em branco antes), com título numerado, em negrito e em letras maiúsculas. Os demais níveis devem ser numerados de forma hierárquica, herdando a numeração da seção onde estão inseridos e acrescentando sua própria numeração, separada da anterior por ponto. Esses níveis devem seguir a seguinte formatação:
\begin{itemize}
	\item seções secundárias: título em maiúsculas;
	\item seções terciárias: título em negrito, tendo somente a primeira letra em maiúsculas;
	\item seções quaternárias: título em itálico, tendo somente a primeira letra em maiúsculas;
	\item seções quinárias: título com somente a primeira letra em maiúsculas.
\end{itemize}

Não devem ser empregadas subdivisões além do quinto nível.

Os elementos pré-textuais como Lista de Figuras, Resumo, Sumário, etc., devem apresentar título sem numeração, centralizado na página, usando a mesma formatação dos capítulos.  O mesmo vale para a lista de Referências Bibliográficas, para os Apêndices e para os Anexos.

\subsection{Exemplo de seção terciária}
Este é um exemplo de seção terciária.

\subsubsection{Exemplo de seção quaternária}
Este é um exemplo de seção quaternária.

\paragraph{Exemplo de seção quinária}
Este é um exemplo de seção quinária.

\section{Listas de itens, listas numeradas e alíneas}
Para subdivisões dentro das seções do texto, a ABNT prevê somente o uso de \emph{alíneas}.  São listas de itens com indicativos progressivos usando as letras minúsculas do alfabeto.  A seguinte lista é um exemplo:
\begin{alineas}
	\item primeira alínea;
	\item segunda alínea;
	\item terceira alínea.
\end{alineas}

É conveniente também, dependendo do caso, utilizar listas numeradas ou de \textit{bullets} (esta última é utilizada em várias partes deste documento).  Em todos os casos, devem ser seguidas algumas observações quanto à formatação:
\begin{enumerate}
	\item os itens devem apresentar recuo de 0,5~cm da margem esquerda do texto;
	\item o texto de cada item deve iniciar por letra minúscula e terminar em ponto-e-vírgula, exceto o último item, que deve terminar em ponto final;
	\item quando o texto de um item ocupar mais de uma linha, as linhas adicionais devem iniciar sob a primeira letra do texto da primeira linha.
\end{enumerate}

\section{Figuras e Tabelas}
O documento pode incluir ilustrações como figuras e tabelas, que devem ser inseridos o mais próximo possível do ponto em que são referenciadas no texto, podendo ser deslocadas para o topo ou a base da página quando conveniente (ex.: para não interromper um fluxo de ideias no texto).  Cada ilustração deve apresentar legenda própria com numeração e descrição de seu conteúdo.  A legenda deve ser posicionada na parte superior da figura ou tabela.  A parte inferior é reservada para indicação de fonte, que deve ser apresentada mesmo que o recurso em questão seja de autoria própria.  A Figura~\ref{fig:escrita} é um exemplo de ilustração cujo conteúdo foi obtido de outro autor.  Já a Tabela~\ref{tab:estacoes} ilustra um exemplo de autoria própria.  

\begin{figure}[h]
	\caption{``Síndrome da tela branca'' na escrita científica}
	\label{fig:escrita}
	\centering%
	\begin{minipage}{.8\textwidth}
		\includegraphics[width=\textwidth]{escrita}
		\fonte{\citet*{Cham12}}
	\end{minipage}
\end{figure}

\begin{table}[h]
	\caption{Períodos das estações do ano no Brasil}
	\label{tab:estacoes}
	\centering%
	\begin{minipage}{.6\textwidth}
		\begin{tabular*}{\textwidth}{ll}
			\hline
			\textbf{Meses} & \textbf{Estações do Ano}\\
			\hline
			21 de março a 21 de junho & Outono\\
			21 de junho a 23 de setembro & Inverno\\
			23 de setembro a 21 de dezembro & Primavera\\
			21 de dezembro a 21 de março & Verão\\
			\hline
		\end{tabular*}
		\fonte{Elaborada pel(o)a autor(a).}
	\end{minipage}
\end{table}

Para cada tipo de ilustração, deve ser apresentada uma lista no início do documento, dentre os chamados elementos pré-textuais.

\section{Notas de rodapé}
Observações breves no decorrer do texto podem ser indicadas por notas de rodapé.  Estas devem ser numeradas progressivamente, colocando-se o indicativo numérico de forma sobrescrita no ponto do texto em que se deseja remeter à nota\footnote{Consulte a NBR~14724 para saber mais detalhes.}.  A nota em si deve ser apresentada na parte inferior da página, separada do texto por um filete de 5cm, destacando à esquerda o expoente numérico correspondente.

\section{Referências bibliográficas}
As referências bibliográficas devem ser apresentadas em lista própria ao final do documento, logo após o último capítulo, mas antes de eventuais apêndices e/ou anexos.  Cada referência deve ser formatada com alinhamento à esquerda (não ``justificado''), em espaçamento simples entre linhas, e separada da anterior por uma linha em branco.

A seguir são apresentados exemplos dos tipos mais comuns de referências bibliográficas.  Para outros exemplos, consulte a NBR~6023 \cite{NBR6023:2002}.

\begin{flushleft}
\subsection{Livros}
%\nocite{Stallings14}
\begin{list}{}
	\item STALLINGS, W.;
  BROWN, L\@. \emph{Segurança de computadores}: princípios e práticas.
  2.~ed. Rio de Janeiro: Elsevier, 2014. ISBN 978-85-352-6449-4.
\end{list}

\subsection{Artigos em periódicos}
%\nocite{Hayes08}
\begin{list}{}
	\item HAYES, B\@. Cloud computing.
  \emph{Communications of the ACM}, New York, v.~51, n.~7, p.~9--11, July~2008.
\end{list}

\subsection{Artigos em eventos}
%\nocite{Anderson95, Borba12b}
\begin{list}{}
	\item ANDERSON, T.~E.
  et~al\@. Serverless network file systems. In: SYMPOSIUM ON OPERATING SYSTEMS
  PRINCIPLES, 15., 1995, Copper Mountain Resort, Colorado.
  \emph{Proceedings{\ldots}} [S.l.:~s.n.], 1995. p.~109--126.
  	\item BORBA, M.~de;
  {\'A}VILA, R.~B\@. Acesso gratuito a {I}nternet - uma proposta de cadastro e
  autentica{\c{c}}{\~a}o para acesso {\`a} {I}nternet em locais p{\'u}blicos.
  In: ESCOLA REGIONAL DE REDES DE COMPUTADORES, ERRC, 10., 2012, Pelotas, RS
 \@. \emph{Anais{\ldots}} [S.l.:~s.n.], 2012.
\end{list}

\subsection{Trabalhos de conclusão de curso, dissertações e teses}
%\nocite{Teixeira09, Flaumann05}
\begin{list}{}
	\item TEIXEIRA, M.~A\@.
  \emph{lattes2latex}: uma ferramenta para conversão de currículos {L}attes
  em documentos {L}a{T}e{X}. 2009. 82~p. Trabalho de Conclus{\~a}o de Curso
  (Bacharelado em Ci{\^e}ncia da Computa{\c{c}}{\~a}o) --- Instituto de
  Inform{\'a}tica, Universidade Federal do Rio Grande do Sul, Porto Alegre,
  2009.\\[4ex]
  	\item FLAUMANN, F.~G\@. \emph{Uma
  proposta de distribuição do servidor de {GNU}s em clusters}. 2005. 125~p.
  Tese (Doutorado em Ci{\^e}ncia da Computa{\c{c}}{\~a}o) --- Programa de
  Pós-Graduação em Computação, UFRGS, Porto~Alegre, 2005.
\end{list}

\subsection{Relatórios técnicos, manuais e normas}
%\nocite{Hexsel04, NBR14724:2011}
\begin{list}{}
	\item HEXSEL, R.~A\@. \emph{Pequeno manual da
  escrita técnica}. Curitiba: Departamento de Informática, Universidade
  Federal do Paraná, 2004. Dispon{\'\i}vel em:
  $<$http://www.inf.ufpr.br/pos/techreport/RT\_DINF004\_2004.pdf$>$. Acesso em:
  set.~2015. (RT-DINF~004/2004).\\[4ex]
  	\item \MakeUppercase{Associação Brasileira de Normas Técnicas}.
  \emph{{NBR}~14724}: informação e documentação -- trabalhos acadêmicos --
  apresentação. Rio de Janeiro, 2011.
\end{list}

\subsection{Documentos publicados na Internet}
%\nocite{Moro11}
\begin{list}{}
	\item MORO, M\@. \emph{A arte de escrever artigos
  cient{\'\i}ficos}. Dispon{\'\i}vel em:
  $<$http://homepages.dcc.ufmg.br/{\char`~}mirella/doku.php?id=escrita$>$.
  Acesso em: set.~2015.
\end{list}
\end{flushleft}

\section{Citações}
As citações a outras obras podem aparecer de forma direta, quando é utilizado literalmente o texto do autor original, ou indireta, quando o texto é próprio porém baseado nas ideias de outro.

Citações diretas de até 3~linhas podem ser apresentadas diretamente no corpo do texto, colocadas entre aspas.  Quando a citação ocupar mais de 3~linhas, ela deve ser destacada do texto com um recuo de 4~cm da margem esquerda, formatada em fonte tamanho~10 e com espaçamento simples.  Como exemplo, segue uma citação de um dos trabalhos referenciados neste documento:
\begin{quote}
Muito se ouve sobre a dificuldade que as pessoas tem para escrever. A dificuldade é real porque escrever demanda esforço e trabalho. Contudo, escrever é uma habilidade que se adquire com a prática, que se aprende. O aprendizado se inicia pela leitura de muitos livros técnicos e artigos, e progride com o exercício. \cite{Hexsel04}
\end{quote}

Toda citação deve ser acompanhada de uma chamada ao trabalho original, que deve possuir item correspondente na lista de referências ao final do trabalho.  A chamada é normalmente feita apresentando o(s) sobrenome(s) do(s) autor(es) e o ano da publicação, podendo ser incluída no texto de duas formas: 
\begin{itemize}
	\item ao final da citação, colocando os sobrenomes dos autores em letras maiúsculas e o ano, ambos entre parênteses.  Exemplo:\\[2ex]
	As técnicas de criptografia são fortemente baseadas na aritmética modular \cite{Stallings14}.
	\item utilizando os sobrenomes dos autores como parte do texto.  Exemplo:\\[2ex]
	Segundo \citet*{Stallings14}, as técnicas de criptografia são fortemente baseadas na aritmética modular.
\end{itemize}

\section{Elementos pré-textuais}
Os elementos pré-textuais precedem a parte principal do texto, localizando-se antes do primeiro capítulo.  Esses elementos devem aparecer na seguinte ordem:
\begin{itemize}
	\item capa;
	\item folha de rosto;
	\item dedicatória (opcional);
	\item agradecimentos (opcional);
	\item resumo em português;
	\item resumo em inglês;
	\item listas de figuras, tabelas, símbolos, etc. (somente os que ocorrerem  no texto);
	\item sumário.
\end{itemize}

Veja no início deste documento quais informações apresentar e como formatar cada um dos itens.

\bibliography{exemplo-tcc}
\end{document}

%=======================================================================
\chapter{Introdução}

% as epígrafes nos capítulos são opcionais
\epigrafecap{The reasonable man adapts himself to the world; the unreasonable one persists in trying to adapt the world to himself. Therefore all progress depends on the unreasonable man.}{George Bernard Shaw}

Conforme \citet*{Hexsel11}, a introdução tem o objetivo de ``\emph{introduzir} o material que vai ser apresentado em mais detalhe nas seções subseqüentes''. Na introdução você deve contextualizar o problema e mostrar por que vale a pena resolvê-lo. Você deve apresentar a solução proposta e mostrar o seu diferencial em relação aos trabalhos relacionados. Observe, porém, que na introdução você deve apenas tratar do O QUÊ e PORQUÊ, sem tratar do como \cite{Hexsel11}, que deve ser explicado na seção que descreve o trabalho desenvolvido.

Geralmente, a introdução tem uma estrutura similar ao resumo e deve apresentar:
\begin{itemize}
	\item \textbf{Contexto e motivação:} Aqui você deve apresentar o contexto do trabalho (área de que ele se trata) e uma motivação para trabalhar nesse assunto.
	\item \textbf{Problema:} Aqui você vai apresentar um problema, uma lacuna, observada na área e que você pretende tratar. Você deve se perguntar aqui: ``Que respostas estou disposto a responder?''. O problema deve ser definido claramente e delimitado em termos de espaço de tempo. Veja que essa parte visa alertar o leitor de que o que você está propondo é uma solução para um problema observado na área. 
	\item \textbf{Objetivos:} Aqui você deve apresentar os objetivos do seu trabalho. Tome cuidado para não confundir objetivos com atividades.   Faça a si mesmo a pergunta: ``O que pretendo alcançar com a pesquisa?''. Você pode discernir entre objetivos gerais e objetivos específicos:
	\begin{itemize}
		\item Objetivo geral --- qual o propósito da pesquisa?
		\item Objetivos específicos --- abertura do objetivo geral em outros menores (possíveis capítulos).
	\end{itemize}
	Veja abaixo um exemplo de objetivo retirado da monografia de~\citet*{Teixeira09}:

	Com a possibilidade de acesso a base de dados XML gerada a partir do Sistema de Currículos Lattes e a necessidade de melhor reutilizar as informações existentes neste sistema, o presente trabalho tem como objetivo geral permitir o acesso do pesquisador a seus dados através de uma interface mais amigável: o padrão LaTeX. Para isto destacam-se os seguintes objetivos específicos:
	\begin{alineas}
		\item identificar e analisar o formato de especificação de currículos da Plataforma Lattes;
		\item disponibilizar uma ferramenta para a geração de uma representação de dados intermediária a partir do formato especificado;
		\item implementar a tradução dos dados colhidos em código LaTeX através da utilização da ferramenta criada;
		\item analisar os resultados obtidos e as alternativas presentes no uso da ferramenta.
	\end{alineas}
\end{itemize}

%=======================================================================
% Escrevendo o Texto
%=======================================================================
\chapter{Escrevendo o Texto}

\section{Comandos do \LaTeX}
Como regra geral, use os comandos tradicionais do \LaTeX\ para formatar seu texto.  Neste documento procuramos demonstrar os comandos mais comumente utilizados em monografias acadêmicas.

Neste capítulo apresentamos alguns exemplos de como colocar figuras e tabelas no seu texto.

\section{Ilustrações}

\subsection{Legendas}
As legendas das figuras devem se encontrar no topo da figura e não abaixo, como usualmente colocado. Abaixo da figura, é obrigatório colocar a fonte (mesmo que a figura tenha sido do próprio autor).

As legendas devem conter o tipo da ilustração (Figura, Tabela, etc), seguido de numeração simples (sem número do capítulo).

Toda figura deve ser citada no texto, como nos exemplos que seguem.

\subsection{Figuras}
A Figura~\ref{fig:escrita} ilustra as fases psicológicas da escrita da dissertação. Você vai se reconhecer no personagem. ;-)


\subsection{Tabelas}
A Tabela~\ref{tab:estacoes} é um exemplo de tabela elaborada pelo(a) próprio(a) autor(a).


\section{Resumo}
O resumo deve conter de 100 a 500 palavras. No resumo não deve haver citações e indica-se que essa seja a última seção do texto a ser escrita. Veja abaixo uma sugestão de organização e exemplo de resumo de \citet*{Moro11}.

Sugestão (uma a três linhas para cada item):
\begin{itemize}
	\item Contexto geral e específico;
	\item Questão/problema sendo investigado (propósito do trabalho);
	\item Estado-da-arte (por que precisa de uma solução nova/melhor);
	\item Solução (nome da proposta, metodologia básica sem detalhes, quais características respondem as questões iniciais, interpretação dos resultados, conclusões).
\end{itemize}

Exemplo (SANTOS et al., 2008 apud \citealp{Moro11}):
\begin{quote}
CONTEXTO: A Web é abundante em páginas que armazenam  dados de forma implícita. PROBLEMA: Em muitos casos, estes dados estão presentes em textos semiestruturados sem a presença de delimitadores explícitos e organizados em uma estrutura também implícita. SOLUÇÃO: Este artigo apresenta uma nova abordagem para extração em textos semi-estruturados baseada em Modelos de Markov Ocultos (Hidden Markov Models - HMM). ESTADO-DA-ARTE e MÉTODO PROPOSTO: Ao contrário de outros trabalhos baseados em HMM, a abordagem proposta dá ênfase à extração de metadados, além dos dados propriamente ditos. Esta abordagem consiste no uso de uma estrutura aninhada de HMMs, onde um HMM principal identifica os atributos no texto e HMMs internos, um para cada atributo, identificam os dados e metadados. Os HMMs são gerados a partir de um treinamento com uma fração de amostras da base a ser extraída. RESULTADOS: Os experimentos realizados com anúncios de classificados retirados da Web mostram que o processo de extração alcança qualidade acima de 0,97 com a medida F, mesmo se esta fração de treinamento é pequena. 
\end{quote}

%=======================================================================
% Exemplos de Citações e Referências Bibliográficas
%=======================================================================
\chapter{Exemplos de Citações e Referências Bibliográficas}
\nobibliography* % para usar o \bibentry
Neste capítulo são apresentados exemplos de citações e referências bibliográficas.  Aqui é utilizado o pacote \texttt{bibentry}, que permite a inserção de referências no meio do texto (atenção para a diferença entre citações e referências).


Em caso de dúvida, siga as orientações do manual da Biblioteca \cite{Biblioteca11} e, se necessário, da norma NBR~6023 \cite{NBR6023:2002}.

\section{Citações}
As citações podem ocorrer de duas formas: com os nomes dos autores inseridos no texto ou não.  Isso implica em uma construção diferente para as frases.  Por exemplo:
\begin{itemize}
	\item Com o nome do autor inserido no texto: ``De acordo com \citet*{Tanenbaum03}, o modelo de referência OSI foi proposto de forma tardia.''
	\item Sem inserir o autor no texto: ``O modelo de referência OSI foi proposto de forma tardia \cite{Tanenbaum03}.''
\end{itemize}

\section{Livros}
Seguem alguns exemplos de referências de livros:
\begin{itemize}
	\item \bibentry{Buford09}.
	\item Livro com indicação de edição:\\
	\bibentry{Kurose10ptbr}.
\end{itemize}

\section{Artigos em Periódicos}
Os exemplos abaixo ilustram referências a artigos em periódicos.
\begin{itemize}
	\item \bibentry{Hayes08}.
	\item \bibentry{Lawton08}.
\end{itemize}

\section{Artigos em Conferências}
\begin{itemize}
	\item \bibentry{Laadan10}.
	\item \bibentry{Anderson95}.
\end{itemize}

\section{Teses e Dissertações}
Seguem algumas referências a trabalhos acadêmicos, como teses, dissertações, trabalhos de conclusão de curso, etc.
\begin{itemize}
	\item \bibentry{Teixeira09}.
	\item \bibentry{Flaumann05}.
\end{itemize}

%=======================================================================
% Referências
%=======================================================================

%=======================================================================
% Exemplo de Apêndice
% O Apêndice é utilizado para apresentar material complementar elaborado
% pelo próprio autor.  Deve seguir as mesmas regras de formatação do
% corpo principal do documento.
%=======================================================================
\appendix
\chapter{Informações Complementares}

O Apêndice é o lugar para incluir textos complementares, que não são essenciais para o entendimento do assunto principal da monografia, mas que podem contribuir com informação relevante (por exemplo, uma prova matemática, uma conceituação básica, etc.).  Ele deve seguir o formato normal do documento.

%=======================================================================
% Exemplo de Anexo
% O Anexo é utilizado para a ``inclusão de materiais não elaborados pelo
% próprio autor, como cópias de artigos, manuais, folders, balancetes, etc.
% e não precisam estar em conformidade com o modelo''.
%=======================================================================
\annex
\chapter{Artigos Publicados}
Existe diferença entre os Apêndices e os Anexos.  Os apêndices trazem informação escrita pelo próprio autor do trabalho, incorporando-se ao formato da monografia como um todo.  Já um anexo é um material à parte, definido/publicado por si só, e que o autor julga conveniente ser apresentado juntamente com a monografia.  Normalmente também vai apresentar formato próprio, como um artigo publicado, um folder, uma planilha, etc.
\end{document}
