%
% Exemplo LaTeX de TCC do IFSul
%
% Este documento é um exemplo de uso do modelo LaTeX de TCCs do IFSul.
% O conteúdo do documento é baseado nas normas ABNT relacionadas à
% elaboração de trabalhos acadêmicos e no Guia de Normalização do
% Instituto (http://www.pelotas.ifsul.edu.br/portal/-
% index.php?option=com_content&view=article&id=1181&Itemid=178).
%
% Os elementos textuais abaixo são apresentados na ordem em que devem
% aparecer no documento.  Repare que nem todos são obrigatórios - isso
% é devidamente indicado em cada caso.  Para usá-los, basta remover
% as marcas de comentário ('%') do início da linha.
%
% Este documento é de domínio público.
%
\documentclass{ifsultcc}

%=======================================================================
% Dados gerais sobre o trabalho.
%=======================================================================
\author{Tal, Fulano de}
\title{Orientações para Elaboração de Trabalhos de Conclusão de Curso
do IFSul Campus Sapiranga}
\orientador{Knuth, Prof.~Dr.~Donald E.}
%\coorientador{Lamport, Prof.~Dr.~Leslie}	% opcional
%\campus{Sapiranga} % opcional
%\local{Sapiranga} % opcional
\curso{Técnico em Manutenção e Suporte em Informática}
\natureza{Trabalho de Conclusão de Curso apresentado como
	requisito parcial para a obtenção do título de
	Técnico em Informática	pelo\\Instituto Federal Sul-Rio-Grandense.\\
	Área de concentração: Informática.}
\ano{2015}
\palavrachave{TCC}
\palavrachave{Modelo}
\palavrachave{Formatação de documentos}
\palavrachave{ABNT}
%\palavrachave[english]{TCC}		% inclua estas se for usado
%\palavrachave[english]{Template}	% o resumo em inglês
%\palavrachave[english]{Document typesetting}
%\palavrachave[english]{ABNT}

%=======================================================================
% Início do documento.
%=======================================================================
\begin{document}
\capa
\folhaderosto

%=======================================================================
% Dedicatória (opcional).
%=======================================================================
%\begin{dedicatoria}
%Aos nossos pais.
%
%\textit{If I have seen farther than others,\\
%it is because I stood on the shoulders of giants.}
%
%--- \textsc{Sir Isaac Newton} % \textsc é o "small caps"
%\end{dedicatoria}

%=======================================================================
% Agradecimentos (opcional).
%=======================================================================
%\begin{agradecimentos}
%Obrigado!
%
%\end{agradecimentos}

%=======================================================================
% Epígrafe do início do documento (opcional).
%=======================================================================
%\begin{epigrafe}
%``\textit{Ninguém abre um livro sem que aprenda alguma coisa}''.\\
%(Anônimo)
%\end{epigrafe}

%=======================================================================
% Resumo em Português.
%
% A recomendação é para 150 a 500 palavras.
%=======================================================================
\begin{abstract}
Este documento apresenta as diretrizes para formatação de TCCs para o IFSul, Campus Sapiranga, elaboradas com base nas normas da ABNT e no Guia de Normalização do IFSul.  São apresentadas as regras de estruturação e formatação gráfica a serem observadas na elaboração dos trabalhos, bem como algumas orientações sobre a escrita do texto em si.  O documento serve também como exemplo de aplicação dessas diretrizes --- todos os elementos de texto aqui apresentados estão formatados de acordo com as normas.  A ABNT recomenda que o Resumo, no caso de trabalhos acadêmicos, contenha de 150 a 500~palavras, e que seja utilizado parágrafo único. Deve-se usar o verbo na voz ativa e na terceira pessoa do singular.
\end{abstract}

%=======================================================================
% Resumo em inglês (obrigatório somente para teses e dissertações).
%=======================================================================
%\begin{otherlanguage}{english}
%\begin{abstract}
%This document presents guidelines for the typesetting of TCC's in IFSul, Sapiranga Campus,  taking as a basis the ABNT Standards and IFSul's ``Guia de Normalização'' (Normalization Guide).  The document presents structuring and graphical formatting rules that must be observed when producing the TCC's, as well as recommendations on the writing of the text itself.  This guide also serves the purpose of being an example on how to apply such rules---every part of this text is presented in accordance to the standard.  ABNT recommends that the Abstract, in the case of academic work, should be composed of 150 to 500~words and that a single paragraph be used.  The text must be written in active voice and using the third person.
%\end{abstract}
%\end{otherlanguage}

%=======================================================================
% Lista de Figuras (somente se houver figuras no texto).
%=======================================================================
\listoffigures

%=======================================================================
% Lista de Tabelas (somente se houver tabelas no texto).
%=======================================================================
\listoftables

%=======================================================================
% Lista de Abreviaturas e Siglas (opcional).
%
% Deve ser passada como parâmetro a maior das abreviaturas ou siglas
% utilizadas.
%=======================================================================
%\begin{listadeabrevsiglas}{FAPERGS}
%\item[ABNT] Associação Brasileira de Normas Técnicas
%\item[ampl.] ampliado, -a
%\item[atual.] atualizado, -a
%\item[CAPES] Coordenação de Aperfeiçoamento de Pessoal de Nível Superior
%\item[coord.] coordenador
%\item[FAPERGS] Fundação de Amparo à Pesquisa do Estado do Rio Grande do Sul
%\item[N.~T.] Novo Testamento
%\item[seg., segs.] seguinte, -s
%\end{listadeabrevsiglas}

%=======================================================================
% Lista de Símbolos (opcional).
%
% Deve ser passado o maior (mais largo) dos símbolos utilizados.
%=======================================================================
%\begin{listadesimbolos}{Ca}
%\item[\textsuperscript{o}C] Graus Celsius
%\item[Al] Alumínio
%\item[Ca] Cálcio
%\end{listadesimbolos}

%=======================================================================
% Sumário
%=======================================================================
\tableofcontents

%=======================================================================
\chapter{Apresentação Gráfica}
Este capítulo relaciona as regras para apresentação gráfica do documento.

\section{Papel e impressão}
Os trabalhos devem ser formatados em tamanho~A4 (210~mm $\times$ 297~mm).  Quando impressos, devem ser usados frente e verso das folhas.  As margens para o texto principal devem ser de:
\begin{itemize}
	\item 3~cm na borda superior;
	\item 2~cm na borda inferior;
	\item 3~cm na borda interna (lado esquerdo nas páginas ímpares e lado direito nas páginas pares);
	\item 2~cm na borda externa (lado direito nas páginas ímpares e lado esquerdo nas páginas pares).
\end{itemize}
  
\section{Formatação geral de texto}
O texto em geral deve ser formatado em fonte Times ou equivalente (ex.: Times New Roman) tamanho~12, com espaçamento 1,5 entre linhas.  Os parágrafos devem apresentar, na primeira linha, um recuo de 0,6~cm da margem esquerda.  Não deve haver espaçamento adicional entre os parágrafos.

Alguns itens de texto devem ser formatados em fonte menor, tamanho~10, e com espaçamento simples.  São eles:
\begin{itemize}
	\item legendas de figuras e tabelas;
	\item notas de rodapé;
	\item citações longas;
	\item numeração das páginas.
\end{itemize}

A lista de referências bibliográficas deve ser formatada com a fonte em tamanho~12, mas com espaçamento simples.

Palavras e expressões em língua estrangeira devem ser apresentadas em destaque, usando o estilo itálico, como em \textit{capability}.

\section{Numeração de páginas}
O número da página deve ser posicionado no topo da folha, a 2~cm da borda superior do papel, e alinhado à margem externa (à direita nas páginas ímpares e à esquerda nas páginas pares).  A numeração deve ser mostrada somente a partir do primeiro capítulo do texto.  As páginas devem ser contadas a partir da folha de rosto (ou seja, a capa não entra na contagem).  

\section{Estruturação do texto}
O texto pode ser dividido em até 5 níveis de seções. As seções primárias (capítulos) devem iniciar em nova folha (sempre em página ímpar --- se necessário, deixe uma página em branco antes), com título numerado, em negrito e em letras maiúsculas. Os demais níveis devem ser numerados de forma hierárquica, herdando a numeração da seção onde estão inseridos e acrescentando sua própria numeração, separada da anterior por ponto. Esses níveis devem seguir a seguinte formatação:
\begin{itemize}
	\item seções secundárias: título em maiúsculas;
	\item seções terciárias: título em negrito, tendo somente a primeira letra em maiúsculas;
	\item seções quaternárias: título em itálico, tendo somente a primeira letra em maiúsculas;
	\item seções quinárias: título com somente a primeira letra em maiúsculas.
\end{itemize}

Não devem ser empregadas subdivisões além do quinto nível.

Os elementos pré-textuais como Lista de Figuras, Resumo, Sumário, etc., devem apresentar título sem numeração, centralizado na página, usando a mesma formatação dos capítulos.  O mesmo vale para a lista de Referências Bibliográficas, para os Apêndices e para os Anexos.

\subsection{Exemplo de seção terciária}
Este é um exemplo de seção terciária.

\subsubsection{Exemplo de seção quaternária}
Este é um exemplo de seção quaternária.

\paragraph{Exemplo de seção quinária}
Este é um exemplo de seção quinária.

\section{Listas de itens, listas numeradas e alíneas}
Para subdivisões dentro das seções do texto, a ABNT prevê somente o uso de \emph{alíneas}.  São listas de itens com indicativos progressivos usando as letras minúsculas do alfabeto.  A seguinte lista é um exemplo:
\begin{alineas}
	\item primeira alínea;
	\item segunda alínea;
	\item terceira alínea.
\end{alineas}

É conveniente também, dependendo do caso, utilizar listas numeradas ou de \textit{bullets} (esta última é utilizada em várias partes deste documento).  Em todos os casos, devem ser seguidas algumas observações quanto à formatação:
\begin{enumerate}
	\item os itens devem apresentar recuo de 0,5~cm da margem esquerda do texto;
	\item o texto de cada item deve iniciar por letra minúscula e terminar em ponto-e-vírgula, exceto o último item, que deve terminar em ponto final;
	\item quando o texto de um item ocupar mais de uma linha, as linhas adicionais devem iniciar sob a primeira letra do texto da primeira linha.
\end{enumerate}

\section{Figuras e Tabelas}
O documento pode incluir ilustrações como figuras e tabelas, que devem ser inseridos o mais próximo possível do ponto em que são referenciadas no texto, podendo ser deslocadas para o topo ou a base da página quando conveniente (ex.: para não interromper um fluxo de ideias no texto).  Cada ilustração deve apresentar legenda própria com numeração e descrição de seu conteúdo.  A legenda deve ser posicionada na parte superior da figura ou tabela.  A parte inferior é reservada para indicação de fonte, que deve ser apresentada mesmo que o recurso em questão seja de autoria própria.  A Figura~\ref{fig:escrita} é um exemplo de ilustração cujo conteúdo foi obtido de outro autor.  Já a Tabela~\ref{tab:estacoes} ilustra um exemplo de autoria própria.  

\begin{figure}[ht]
	\caption{``Síndrome da tela branca'' na escrita científica}
	\label{fig:escrita}
	\begin{area}{10cm}
		\includegraphics{escrita.jpg}
		\fonte{\cite{Cham12}}
	\end{area}
\end{figure}

\begin{table}[ht]
	\caption{Períodos das estações do ano no Brasil}
	\label{tab:estacoes}
	\begin{area}{10cm}
		\begin{tabularx}{\textwidth}{Xc}
			\hline
			\textbf{Período} & \textbf{Estação do Ano}\\
			\hline
			21 de março a 21 de junho & Outono\\
			21 de junho a 23 de setembro & Inverno\\
			23 de setembro a 21 de dezembro & Primavera\\
			21 de dezembro a 21 de março & Verão\\
			\hline
		\end{tabularx}
		\fonte{Elaborada pel(o)a autor(a)}
	\end{area}
\end{table}

Para cada tipo de ilustração, deve ser apresentada uma lista no início do documento, dentre os chamados elementos pré-textuais.

\section{Notas de rodapé}
Observações breves no decorrer do texto podem ser indicadas por notas de rodapé.  Estas devem ser numeradas progressivamente, colocando-se o indicativo numérico de forma sobrescrita no ponto do texto em que se deseja remeter à nota\footnote{Consulte a NBR~14724 para saber mais detalhes.}.  A nota em si deve ser apresentada na parte inferior da página, separada do texto por um filete de 5cm, destacando à esquerda o expoente numérico correspondente.

\section{Referências bibliográficas}
As referências bibliográficas devem ser apresentadas em lista própria ao final do documento, logo após o último capítulo, mas antes de eventuais apêndices e/ou anexos.  Cada referência deve ser formatada com alinhamento à esquerda (não ``justificado''), em espaçamento simples entre linhas, e separada da anterior por uma linha em branco.

A seguir são apresentados exemplos dos tipos mais comuns de referências bibliográficas.  Para outros exemplos, consulte a NBR~6023 \cite{NBR6023:2002}.

% o uso de flushleft abaixo é justamente para, nestes exemplos, deixar o texto alinhado
% somente à esquerda
\subsection{Livros}
\begin{flushleft}
	STALLINGS, W.; BROWN, L\@. \emph{Segurança de computadores}: princípios e práticas. 2.~ed. Rio de Janeiro: Elsevier, 2014. ISBN 978-85-352-6449-4.
\end{flushleft}

\subsection{Artigos em periódicos}
\begin{flushleft}
	HAYES, B\@. Cloud computing. \emph{Communications of the ACM}, New York, v.~51, n.~7, p.~9--11, July~2008.
\end{flushleft}

\subsection{Artigos em eventos}
\begin{flushleft}
	ANDERSON, T.~E. et~al. Serverless network file systems. In: SYMPOSIUM ON OPERATING SYSTEMS PRINCIPLES, 15., 1995, Copper Mountain Resort, Colorado. \emph{Proceedings{\ldots}} [S.l.:~s.n.], 1995. p.~109--126.\\[.5cm]

  	BORBA, M.~de; ÁVILA, R.~B\@. Acesso gratuito a Internet -- uma proposta de cadastro e autenticação para acesso à Internet em locais públicos. In: ESCOLA REGIONAL DE REDES DE COMPUTADORES, ERRC, 10., 2012, Pelotas, RS\@.\emph{Anais{\ldots}} [S.l.:~s.n.], 2012.
\end{flushleft}

\subsection{Trabalhos de conclusão de curso, dissertações e teses}
\begin{flushleft}
	TEIXEIRA, M.~A\@. \emph{lattes2latex}: uma ferramenta para conversão de currículos Lattes em documentos LaTeX\@. 2009. 82~p. Trabalho de Conclusão de Curso (Bacharelado em Ciência da Computação) --- Instituto de Informática, Universidade Federal do Rio Grande do Sul, Porto Alegre, 2009.\\[.5cm]
	
	FLAUMANN, F.~G\@. \emph{Uma proposta de distribuição do servidor de GNUs em clusters}. 2005. 125~p.  Tese (Doutorado em Ciência da Computação) --- Programa de Pós-Graduação em Computação, UFRGS, Porto~Alegre, 2005.
\end{flushleft}

\subsection{Relatórios técnicos, manuais e normas}
\begin{flushleft}
	HEXSEL, R.~A\@. \emph{Pequeno manual da escrita técnica}. Curitiba: Departamento de Informática, Universidade Federal do Paraná, 2004. Disponível em: $<$http://www.inf.ufpr.br/pos/techreport/RT\_DINF004\_2004.pdf$>$. Acesso em: set.~2015. (RT-DINF~004/2004).\\[.5cm]
	
	ASSOCIAÇÃO BRASILEIRA DE NORMAS TÉCNICAS. \emph{{NBR}~14724}: informação e documentação -- trabalhos acadêmicos -- apresentação. Rio de Janeiro, 2011.
\end{flushleft}

\subsection{Documentos publicados na Internet}
\begin{flushleft}
	MORO, M\@. \emph{A arte de escrever artigos científicos}. Disponível em:
  $<$http://homepages.dcc.ufmg.br/{\char`~}mirella/doku.php?id=escrita$>$.  Acesso em: set.~2015.
\end{flushleft}

\section{Citações}
As citações a outras obras podem aparecer de forma direta, quando é utilizado literalmente o texto do autor original, ou indireta, quando o texto é próprio porém baseado nas ideias de outro.

Citações diretas de até 3~linhas podem ser apresentadas diretamente no corpo do texto, colocadas entre aspas.  Quando a citação ocupar mais de 3~linhas, ela deve ser destacada do texto com um recuo de 4~cm da margem esquerda, formatada em fonte tamanho~10 e com espaçamento simples.  Como exemplo, segue uma citação de um dos trabalhos referenciados neste documento:
\begin{quote}
Muito se ouve sobre a dificuldade que as pessoas tem para escrever. A dificuldade é real porque escrever demanda esforço e trabalho. Contudo, escrever é uma habilidade que se adquire com a prática, que se aprende. O aprendizado se inicia pela leitura de muitos livros técnicos e artigos, e progride com o exercício. \cite{Hexsel04}
\end{quote}

Toda citação deve ser acompanhada de uma chamada ao trabalho original, que deve possuir item correspondente na lista de referências ao final do trabalho.  A chamada é normalmente feita apresentando o(s) sobrenome(s) do(s) autor(es) e o ano da publicação, podendo ser incluída no texto de duas formas: 
\begin{itemize}
	\item ao final da citação, colocando os sobrenomes dos autores em letras maiúsculas e o ano, ambos entre parênteses.  Exemplo:\\[2ex]
	As técnicas de criptografia são fortemente baseadas na aritmética modular \cite{Stallings14}.
	\item utilizando os sobrenomes dos autores como parte do texto.  Exemplo:\\[2ex]
	Segundo \citet*{Stallings14}, as técnicas de criptografia são fortemente baseadas na aritmética modular.
\end{itemize}

\section{Elementos pré-textuais}
Os elementos pré-textuais precedem a parte principal do texto, localizando-se antes do primeiro capítulo.  Esses elementos devem aparecer na seguinte ordem:
\begin{itemize}
	\item capa;
	\item folha de rosto;
	\item dedicatória (opcional);
	\item agradecimentos (opcional);
	\item resumo em português;
	\item resumo em inglês;
	\item listas de figuras, tabelas, símbolos, etc. (somente os que ocorrerem  no texto);
	\item sumário.
\end{itemize}

Veja no início deste documento quais informações apresentar e como formatar cada um dos itens. 

%=======================================================================
% Referências bibliográficas
%=======================================================================
\bibliography{exemplo-tcc}

%=======================================================================
% Apêndice
%=======================================================================
%\appendix
%\chapter{Informações Complementares}
%
%O Apêndice é o lugar para incluir textos complementares, que não são essenciais para o entendimento do assunto principal da monografia, mas que podem contribuir com informação relevante (por exemplo, uma prova matemática, uma conceituação básica, etc.).  Ele deve seguir o formato normal do documento.
%
%Existe diferença entre os Apêndices e os Anexos.  Os apêndices trazem informação escrita pelo próprio autor do trabalho, incorporando-se ao formato da monografia como um todo.  Já um anexo é um material à parte, definido/publicado por si só, e que o autor julga conveniente ser apresentado juntamente com a monografia.  Normalmente também vai apresentar formato próprio, como um artigo publicado, um folder, uma planilha, etc.

%=======================================================================
% Anexos
%=======================================================================
%\annex
%\chapter{Artigos Publicados}
%\noindent (neste exemplo, o autor anexaria fisicamente a este documento --- ou seja, com cópias impressas --- os artigos que publicou ao longo do trabalho.)

\end{document}
